\section{Background}\label{sec:back}

Innholdsfortegnelse:

\begin{itemize}

	\item{Hva er modellsjekking?}
	
	\begin{itemize}
		\item{I en dagligdags kontekst brukes til alt mulig, mer enn bare logikk, økonomi, spillteori, forhandlinger pluss mer}
		\item{Diskutere i mer abstrakt kontekst, glidende overgang fra lister av flyavganger og reiseplanleggere (Kan svare på spørsmål om 'finnes det et ledig fly fra Bergen til Oslo den 16?', til mer strukturerte systemer som databaser, med spørrespråk som SQL hvor vi kan utforme våre spørringer, til faktisk logikk.}
		\item{Logikere flest vil nok alikevel ikke anerkjenne reiseplanleggere (Momondo) som reelle modellsjekkere}
	\end{itemize}
	
	\item{Definere modellsjekking / modellsjekkere}
	
	\item{Består av tre komponenter}
	\begin{itemize}
		\item{En modell som kan beskrive noe vi er interessert i (Database, tabell med flyavganger, logiske strukturer, Kripke)}
		\item{Et språk som kan formulere spørringer eller påstander om denne modellen (SQL, naturlig språk, GAL)}
		\item{En mekanisme som kan besvare spørringene, semantikk, relasjon (Momondo, databaseverktøy, semantikk + tilfredsstillelsesrelasjon, ($\models$)})
	\end{itemize}
	
	\item{Hvorfor modellsjekking?}
	
	\begin{itemize}
	
		\item{Brukes til å bedre vår forståelse av abstrakte strukturer og systemer (Inkludert logikk?)}
		
		\begin{itemize}
			\item{Spillteoretiske modeller brukes av økonomer til å forstå og utforske markedsstrategier}
			\item{Lang tradisjon av bruk i computer science til formell verifikasjon av programkode, nyttig i høyrisiko-sammenheng, som styringssystem på romferger, firmware på medisinsk utstyr, militære formål}
			\item{Systemer blir mer komplekse, involverer flere uavhengige agenter i distribuerte systemer, hvordan kommuniserer slike agenter? Finnes det en sekvens med handlinger som gjør at et multi-agent system går i lås (deadlocks)?}
			\item{Modellsjekking kan hjelpe utforske slike problemstillinger}
			\item{Nevne CTL som eksempel på bruk av logikk i slike sammenhenger}
			\item{Nevne AT(E?)L i kontekst av spillteoretiske modeller?}
		\end{itemize}

	\todo{Prøve å finne eksempel på hvorfor DEL/PAL/GAL}

	\end{itemize}

	\item{Nå som nytten av modellsjekking er etablert, hvorfor modellsjekkere?}

	\begin{itemize}
		\item{Stor kompleksitet, urimelig å gjennomføre manuellt, 700 sider med flyavganger, finnes det et ledig fly fra Bergen til Oslo uten mellomlandinger?}
		\item{Eksterne værktøy kan bidra, reiseplanleggere som Momondo, eksekveringsmotorer for spørrespråk som SQL, modellsjekkere for formelle logikker}
		\item{Feltet i dag (State of the art), nevne eksisterende verktøy, DEMO og SmcDEL, påpeke at ingenting for GAL (enda)}
		\item{Hva kan jeg bidra med?}
		\begin{itemize}
			\item{Ingen modellsjekkere for GAL}
			\item{Pedagogisk vinkling, verdi av grafisk verktøy i undervisningskontekst}
			\item{Eksisterende verkøy har dårlige brukergrensesnitt, krever ekspertkompetanse for å ta i bruk, stort forbedringspotensiale}
			\item{Forenkle læringsprosessen gjennom å visualisere abstrakte konsepter på en intuitiv måte}
		\end{itemize}
		
	\end{itemize}
	
\end{itemize}


%"Information is communicated, so knowledge and belief are by no means static. Not surprisingly, many logicians have taken this into account. In the context of epistemic logic, there are many different approaches. Dynamic epistemic logic is an umbrella term for a number of extensions of epistemic logic with dynamic operators that enable us to formalize reasoning about information change. It came forth from developments in formal linguistics, computer science, and philosophical logic."\cite{Ditmarsch2007}


%\subsection{Logic and Kripke models}

%Epistemic logic is a branch of logic concerned with reasoning around the knowledge and beliefs of agents in systems, represented through formal languages and models. Many of the logics within this branch work with structures we call Kripke models. These Kripke models divide the world into three main components, a set of possible worlds, a set propositions which may or 

%These Kripke models model the world around us as a set of possibilities, where each possibility splits the world in two. Either it rains or it doesn't rain, and if you know whether it rains then you can tell these possibilities apart. In these models we also have agents, whose knowledge we represent through their ability to tell these worlds apart. In these worlds certain properties hold, 


%Epistemic logic is a breanch of logic focused on reasoning around knowledge. More specifically on reasoning around the knowledge of agents in our systems. These systems 

The logic this thesis will be working with, group announcement logic, is also one of these extensions of epistemic logic, but in addition to reasoning around information change also allows us to reason around what coalitions of agents are able to achieve through making public announcements in unison based on how they can change the information available to other agents in the system. Group announcement logic, like many other forms of epistemic logic revolve around the notion of so-called `possible worlds' and agents which may or may not be able to distinguish between them. When working with these possible worlds, we usually refer to them as states in some system we are trying to simulate. We describe which agents are able to distinguish between these states based on an accessibility relation for each agent, consisting of pairs of states they are unable to differentiate. Since we are solely working with epistemic models in this thesis, also known as S5 models, these accessibility relations will also be equivalence relations (meaning they are reflexive, symmetric and transitive), and we will from here on also refer to them as such.

These simulations are then grouped into models consisting of a set of possible worlds or states, a set of equivalence relations for each agent we wish to model, as well as a set of boolean propositions which may or may not hold in a given state. Based on this set of possible worlds and these accessibility relations between them we can model what each agent in our system knows by defining knowledge as being 


\subsection{Model checking}

Model checking refers to the act of checking whether or not a formula holds in a given model. A simple example of model checking might be to check if its always true that Alice knows whether or not it is raining outside in the context of some model. A more interesting example could be to check whether or not there exists a sequence of actions that can be taken which might make a distributed system deadlock. Model checking is a general problem not just within knowledge representation and artificial intelligence, but also in other fields of computer science such as formal verification of communication protocols and software correctness.

With model checking utilities being such useful tools for exploring the properties of more complex models, it should come as no wonder that there exists plenty of model checkers out there for various forms of logic. Most traditional model checkers however focus on answering simple yes or no questions in regards to whether something is true or not. One of the closest examples of this is DEMO\_S5 for dynamic epistemic logic (DEL), being a command-line application that lets the user types in their formulas and check it against models formatted as long strings of text. We aim to create a tool which can assist users in understanding how these systems work by providing a graphical editor that can visualize both the models themselves as well as the checking process.


%Mention Tarski's World as example of educational software\\
%Mention DEMO and SMCDEL as examples of model checkers in epistemic logic.\\
%Highlight DEMO's usability problem, shitty UI\\
%Explain how \cname{} will attempt to improve this\\

%We will introduce a model checking tool 
%\todo{Mention Fitch, DEMO, SMCDEL++ and compare in terms of usability and goal}
