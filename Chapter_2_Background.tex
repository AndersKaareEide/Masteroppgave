\section{Background}\label{sec:back}

\subsection{Motivation}

In 2010, Ågotnes et al. published a paper on their extension of Public Announcement Logic called Group Announcement Logic (GAL) \cite{Agotnes2010}. While the group announcement operator introduced in this logic is highly expressive and enables us to model interesting properties of knowledge in groups and the ability of agents to impact the system around them, this expressiveness can also be somewhat confusing and complicated to understand at first. 

For this reason, having a learning aid of some sort, capable of visualizing how the various operators of this logic (and group announcements in particular) work could greatly aid future students and logicians in understanding how GAL and related logics work, especially seeing how other visual tools such as JFLAP\footnote{Available from: \url{http://www.jflap.org/}} have previously been used to great effect in teaching similar topics. 

When we work with these logics, we are commonly trying to assert whether or not our models exhibit some property. While we will return to GAL and define and discuss it later in this thesis, it might be a good idea to start with a discussion of how these logics are typically used. This brings us to the topic of model checking, which is one of the key concepts this thesis will be covering.

\subsection{Model checking}

What is model checking? Phrased abstractly, model checking is a way for us to further our understanding of highly complex systems or structures by formalizing ways of analyzing them. Generally it involves the process of verifying whether or not our model exhibits certain properties. Depending on the type of logic we are working with, our models are commonly represented as Kripke structures\cite{Van2007dynamic} (which we will be using and discussing later in this thesis), but can also be described in more purely algebraic forms. While we use these models to model certain aspects of the world around us, we also require ways of describing its properties, which is where our logics come in. The logics define not only what constitutes a legal formula (i.e. a syntax), but also their meaning (i.e. the semantics of all valid ways of building formulas). 

Model checking has a rich tradition in the field of computer science and has seen a lot of practical use in formal software and hardware verification. A few examples of properties which can be interesting to verify would be liveness (regardless of system state, we can always reach a `reset' state and guarantee that the system will acknowledge and answer an incoming request, i.e. not deadlock) and safety (ensuring that the system can never enter a potentially dangerous state). While model checking is in most cases not a replacement for traditional forms of software and hardware testing, it can be seen as a complementary technique to be used in high-risk scenarios where the cost of failure offsets the investment into utilizing more formal verification methods to ensure correctness.

A more concrete example that also highlights one such high-risk scenario is the case of the Ariane 5 rocket, mentioned by Clarke et. al in their handbook on model checking\cite{Clarke1999}, as an example of why we need formal methods for verifying software and hardware integrity. This example, where the rocket ended up exploding shortly after takeoff due to a conversion error in the software controlling the guidance system, shows just one of many situations where we are forced to rely on systems to carry out tasks of critical importance to us. As these systems grow not just in complexity, but also in importance and their effects on our daily lives, it also becomes increasingly important to improve our methods of verifying the behavior of these systems. 

Traditionally, practical applications of model checking has been dominated by simpler forms of logic such as Computation Tree Logic\cite{clarke1981design} (CTL) and Linear time Temporal Logic\cite{Pnueli1977_LTL} (LTL), whereas more complex logics such as Public Announcement Logic\cite{Plaza2007_PAL} (PAL), Alternating-time Temporal Logic\cite{ATL} (ATL) and Group Announcement Logic tend to be regarded as more `academical pursuits'. While these simpler logics have the obvious advantage of being far simpler to efficiently model check, leading to a wide variety of available tools such as BLAST\cite{BLAST} or SPIN\footnote{Public repository for SPIN can be found at: https://github.com/nimble-code/Spin}\cite{SPIN} (which at the time of writing will soon have its 30th anniversary while still being actively maintained), they are also less expressive than their more modern academic counterparts. However as our systems grow ever more distributed (especially in today's focus on `autonomous' systems and IoT appliances) it might no longer be enough to verify the behavior of our systems in isolation. For example, if we are to guarantee the safety of tomorrow's fully self-driving cars, it could be useful to simulate the actions of these systems as autonomous agents, individually capable of influencing the state of a more complex network of such agents. It is when it comes to modeling more complex situations and networks such as these, that we might require more expressive forms of logic such as for example ATL, its epistemic extension ATEL\cite{ATEL} (Alternating-time Temporal Epistemic Logic), or as we will be exploring, GAL. As these epistemic logics allow us to not only reason around the knowledge of each agent in our systems, but also their knowledge of other agents' knowledge, it makes sense that these epistemic logics could bring much of interest to the field of formal software verification, given enough time, interest and support in the form of powerful model checking tools. Unfortunately however, the complexities of these logics also mean that the problem of automated model checking for them is not sufficiently well explored. Looking back at our previously mentioned logics like CTL and LTL, we believe that the field of model checking dynamic epistemic logics lags far behind, especially in the case of GAL, which does not yet have a model checker written for it at all. 

\subsection{State of the art}

While model checkers do exist for similar kinds of logic such as DEMO\_S5\footnote{Haskell source files and user guides available from: \url{https://homepages.cwi.nl/~jve/software/demo_s5/}} for PAL and SMCDEL\footnote{Public repository found at https://github.com/jrclogic/SMCDEL}\cite{GattingerSMCDELPhD} for DEL (Dynamic Epistemic Logic) most of them tend to be rather hard to use and often require a good deal of time and expert knowledge in order to learn how to use. This, combined with the fact that both of our previously mentioned examples base themselves on using a Haskell REPL (read-eval-print-loop) for user interaction means  they would most likely also lend themselves poorly to teaching. While there does exist some fairly powerful visualization libraries/modules generating visualizations of Kripke structures such as Gattinger's KripkeVis module\footnote{Haskell source files and installation guide available from \url{https://w4eg.de/malvin/illc/kripkevis/}}, they still require a fairly great deal of technical competence in order to set up correctly, and at least in the case of KripkeVis, only produce static visualizations, instead of facilitating direct manipulation of the models.

A key reason for why these more complex logics have not seen much use in practical applications of model checking yet, is that their expressiveness not only leads to increased complexity in terms of the implementation of model checkers, but also in terms of computational complexity of the model checking process itself. While there has been some recent developments in terms of speeding up model checking of epistemic logics by way of symbolic representations such as using binary decision diagrams\cite{BDDs} (BDDs) instead of concrete models for DEL\cite{GattingerSMCDELPhD}, this thesis is more concerned with the creation of an easily approachable tool for learning, rather than for research or practical applications which might require more efficient algorithms.

%\todo{Insert image of DEMO}

This brings us to where we wish to make a contribution to the field. By developing an easy to use model checking tool that can be utilized in an educational context, we hope to spark more interest in these logics and help make them more accessible to others by providing intuitive visualizations of how the logic works. As such we not only wanted to make a model checker that supports group announcement logic, but also make it as intuitive as possible by giving it a graphical user interface that allows the user to interact with their models in a more direct manner than more algebraic formats used by other tools to describe theirs. 

Our reasoning behind this is that by moving away from the Haskell REPLs used by DEMO\_S5 and SMCDEL and crafting an application that has a more intuitive graphical user interface, we could significantly lower the barrier of entry and make our tool far easier to adopt in educational settings, than its counterparts. Additionally, by making it capable of visualizing each step of the checking process, we believe that it can greatly benefit students and fledgling logicians trying to understand how the various operators of GAL and epistemic logic in general work. 

There is also an interesting theoretical challenge involved in model checking GAL, as the semantics behind the logic's group announcement operator involve quantifying over an infinite set of formulas that a given coalition of agents can announce. To this end, we will present an algorithm for enumerating this infinite set of formulas by exploring the properties of minimal bisimilar structures (models stripped of any redundant information) and grouping these formulas by which states in our structures they are satisfied in. 

%The logic this thesis will be working with, group announcement logic, is also one of these extensions of epistemic logic, but in addition to reasoning around information change also allows us to reason around what groups of these agents can accomplish by working together. In the following sections we aim to give the reader some insight into this logic as we work with its definitions towards something something tool.

%While model checking utilities can be incredibly useful tools for exploring the properties of our models, most traditional model checkers focus on answering simple yes or no questions in regards to whether something is true or not. We will present a tool towards the end of this thesis which can assist users in understanding how these systems work by providing a graphical editor that can visualize both the models themselves as well as the checking process.

%Explain how \cname{} will attempt to improve this\\

\subsection{Structure}

This thesis is divided into the following sections: 

\begin{itemize}
	\item{This background section dedicated to introducing the reader to the field of dynamic epistemic logic, discussing the topic of model checking as well as its usages and outlining the niche we aim to fill with our own model checking tool.}
	\item{A section about the logic this thesis will be working with, Group Announcement Logic, where we will be presenting most of the definitions that will be used in the following sections.}
	\item{The next section will contain the majority of our theoretical work behind this paper, where we discuss the semantics of group announcement logic and will explore its definitions in order to rework them into definitions that are more easily translatable into algorithms which we will use in our model checker.}
	\item{We will then translate these definitions into algorithms presented through pseudocode in another section, discussing any differences between the logical semantics and our algorithms.}
	\item{After this, we will present the results of our work in the form of a fully-functional graphical model checking tool for GAL, that also doubles as a teaching aid. Here we will present its various features and our rationales for implementing them.}
	\item{Once we have finished our high-level introduction and presentation of our tool, we will then transition into a more in-depth discussion of how the features from the previous section were implemented, as well as our choices of technologies and libraries and which advantages and disadvantages they brought.}
	\item{Finally, we will recap with a discussion of our results, our experiences during development and what potential future work could be done to improve our tool further.}
\end{itemize}

%Something something one of the many measures taken to prevent similar incidents from occurring during future launches was the application of formal software verification techniques to prevent similar errors.

%Something something, obviously these tasks are not done by hand, which is where tools such as model checkers come in. By automating the process of checking and/or verifying properties of our software we can throw huge amounts of computational power at our exploration of them. In order to benefit from this model checking however, we require precise models, again requiring understanding of how these models work. This is 

%Foo mentions case of Therac-25 machines which had a software bug causing several patients to receive fatal overdoses of radiation upon treatment due to a concurrency error in its software. 


%Group announcement logic, like many other forms of epistemic logic revolve around the notion of so-called `possible worlds' and agents which may or may not be able to distinguish between them. When working with these possible worlds, we usually refer to them as states in some system we are trying to simulate. We describe which agents are able to distinguish between these states based on an accessibility relation for each agent, consisting of pairs of states they are unable to differentiate. Since we are solely working with epistemic models in this thesis, also known as S5 models, these accessibility relations will also be equivalence relations (meaning they are reflexive, symmetric and transitive), and we will from here on also refer to them as such.
%
%These simulations are then grouped into models consisting of a set of possible worlds or states, a set of equivalence relations for each agent we wish to model, as well as a set of boolean propositions which may or may not hold in a given state. Based on this set of possible worlds and these accessibility relations between them we can model what each agent in our system knows by defining knowledge as being 


%Before we delve deeper into the logics and semantics this thesis will be working with, we would like to discuss what makes model checking so useful. 
%Having briefly touched upon what model checking is, it is time to discuss why model checking is needed and what makes it so useful. 

%Model checking is a general problem not just within knowledge representation and artificial intelligence, but also in other fields of computer science such as formal verification of communication protocols and software correctness.
%\begin{itemize}
%	\item{A model, capable of modeling something we are interested in. This could be virtually anything from our previously mentioned flight schedules, databases or more formal models as used in game theory or the Kripke structures we will be using in this thesis.}
%	\item{A language capable of formulating queries or propositions in regards to these models. Examples here includes natural language, more formal query languages such as SQL or various logics such as Public Announcement Logic or Computation Tree Logic.}
%	\item{A mechanism which can answer these queries in some fashion. One such mechanism would be the travel planner from our previous example, the tools that execute our SQL, or in our case, the semantics behind our logic and it's satisfaction relation. }
%\end{itemize}
%	
%
%\todo{
%\\* Discuss different approaches towards coalitional ability in dynamic epistemic logic, mention coalitional logics like ATL?
%\\* Discuss educational benefits from usage of model checker, visualization of semantics behind operators
%\\* Discuss scope of implementation, not suitable for research work, but potentially useful in educational setting
%}

%JFLAP: 
%Started as a set of tools at Rensselaer Polytechnic Institute around 1990, moved to Duke University in 1994
%The creators of JFLAP describe it as "a package of graphical tools which can be used as an aid in learning the basic concepts of Formal Languages and Automata Theory."
%JFLAP introduces the concept of turing machines which can be rather abstract for a fledgeling student to grasp
%While we previously mentioned economists using game theory to analyze decision making in various business scenarios, there is also a long tradition in computer science of using model checking to formally verify that hardware and software meets certain requirements. While these formal verification methods often require major investments in time and resources, they still have their place in high-risk contexts, where the cost of failure is particularly high, or human lives are at stake. 
%
%While the concept of model checking is most frequently associated with logic, if we examine it in an everyday context, aspects of other fields such as game theory in relation to economics or .
%
% most tasks around us involve aspects of this in one way or another. 
%Examined in an everyday context, the concept of model checking is all around us, frequently used 
%	
%	\begin{itemize}
%		\item{I en dagligdags kontekst brukes til alt mulig, mer enn bare logikk, økonomi, spillteori, forhandlinger pluss mer}
%		\item{Diskutere i mer abstrakt kontekst, glidende overgang fra lister av flyavganger og reiseplanleggere (Kan svare på spørsmål om 'finnes det et ledig fly fra Bergen til Oslo den 16?', til mer strukturerte systemer som databaser, med spørrespråk som SQL hvor vi kan utforme våre spørringer, til faktisk logikk.}
%		\item{Logikere flest vil nok alikevel ikke anerkjenne reiseplanleggere (Momondo) som reelle modellsjekkere}
%	\end{itemize}
%	
%As such, we would argue that there exists a `sliding scale' of formality in regards to model checking ranging from lists of plane departures and trip planners, which can be used to answer questions along the lines of `is there a direct flight between Bergen and Oslo departing on the 16th?', to more structured systems such as databases 
%
%However, as most logicians probably would not recognize trip planners such as Momondo or [Insert tool here] as proper model checking tools, we would like to present a somewhat loose and abstract definition of model checking and model checkers. For us, a model checker consists of the following three main components:
%
%	\item{Består av tre komponenter}
%	\begin{itemize}
%		\item{En modell som kan beskrive noe vi er interessert i (Database, tabell med flyavganger, logiske strukturer, Kripke)}
%		\item{Et språk som kan formulere spørringer eller påstander om denne modellen (SQL, naturlig språk, GAL)}
%		\item{En mekanisme som kan besvare spørringene, semantikk, relasjon (Momondo, databaseverktøy, semantikk + tilfredsstillelsesrelasjon, ($\models$)})
%	\end{itemize}
%	
%	\item{For hensyns skyld så er det imidlertid lurt å begrense diskusjonen til mer formelle systemer som kan brukes til noe praktisk}
%"Information is communicated, so knowledge and belief are by no means static. Not surprisingly, many logicians have taken this into account. In the context of epistemic logic, there are many different approaches. Dynamic epistemic logic is an umbrella term for a number of extensions of epistemic logic with dynamic operators that enable us to formalize reasoning about information change. It came forth from developments in formal linguistics, computer science, and philosophical logic."\cite{Ditmarsch2007}


%\subsection{Logic and Kripke models}

%Epistemic logic is a branch of logic concerned with reasoning around the knowledge and beliefs of agents in systems, represented through formal languages and models. Many of the logics within this branch work with structures we call Kripke models. These Kripke models divide the world into three main components, a set of possible worlds, a set propositions which may or 

%These Kripke models model the world around us as a set of possibilities, where each possibility splits the world in two. Either it rains or it doesn't rain, and if you know whether it rains then you can tell these possibilities apart. In these models we also have agents, whose knowledge we represent through their ability to tell these worlds apart. In these worlds certain properties hold, 


%Epistemic logic is a breanch of logic focused on reasoning around knowledge. More specifically on reasoning around the knowledge of agents in our systems. These systems 
