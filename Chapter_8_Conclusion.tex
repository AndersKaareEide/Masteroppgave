\section{Conclusion}\label{sec:impl}

%Recap:
% Presented GAL
% Introduced and discussed its definitions for group announcements
% Presented our own definitions for easier implementation
% Converted these new definitions into pseudocode algorithms
% Implemented these algorithms in our own 

Now that we have covered all of our topics and presented \cname{}, a novel educational tool for teaching Group Announcement Logic, it is time to summarize our contributions.

In this thesis we have presented a concrete set of algorithms for not just model checking group announcements, but also for computing bisimulation and bisimulation contracted models. Additionally we presented our own revised definitions for the group announcement operator, with clearer semantics, making them easier to formulate model checking algorithms for. Additionally, we also present our own revised definition for GAL's group announcement operator as well as a set of algorithms for expressing its semantics through pseudocode. Finally, we presented the implementation of these algorithms in our fully realized model checker, \cname{}, with an intuitive user interface capable of visualizing how each of the GAL's operators work. 

While both user testing and educational impact studies go beyond the scope of this thesis, we believe we have developed a highly useful educational tool that will make it much easier to learn these logics in the future. We also hope that the development of this model checker and for GAL and the concretization of the semantics behind its group announcement operator can help spark continued interest in the logic itself.

% and it would be highly interesting to see someone take up the torch and prove the value of \cname{}. 

\subsection{Future work}

Although \cname{} is fully functional educational aid in its current state, there are also a fair few features that we simply did not have time to implement, which we will discuss here.

One of the simpler additions to the tool is implementing the dual, also known as `diamond' version of the announcement operators. As these operators do not add any additional expressiveness or capabilities to the model checker they were never prioritized as they can always be expressed through the negation of the box operators. It would still be nice to have support for these operators directly however, to cut down on formula length and complexity when visualizing larger formulas. As both the ANTLR grammar and related formula components are easily extendable, implementing these operators would be fairly trivial as their underlying semantics are basically already implemented.

There are also a fair few additions to \cname's UI we would have liked to implement, such as separately displaying the set of formula extensions a coalition can announce or provide the user with more informative error messages when attempting to parse syntactically incorrect formulas. Visualizing these formula extensions should also not prove all too challenging as the extensions are already generated when checking formulas. Similarly, the ANTLR parsers provide most of the context necessary to present inform the user of which part of their string caused an error, although more complex reasoning around how to parse ambiguous structures in regards to how to handle missing parentheses and the like might be more challenging. 

There were also plans for generalizing \cname{}'s model serializer in order to be able to export the models as formats beyond its current basic binary format such as GEXF\footnote{\url{https://gephi.org/gexf/format/}} in order to be able to view these models in other tools such as Gephi\footnote{\url{https://gephi.org/}}. As the intended users of the tool are mainly students attempting to gain a better understanding of the semantics of the operators and structures in group announcement logic however, the feature was eventually scrapped as the models created would likely not be all that interesting to visualize in external tools anyway and the work involved would be fairly substantial for a feature that would probably go unused by most users.  Continuing on model serialization, we would also have liked to be able to store additional information or metadata about each model, such as being able to write notes about interesting properties a model might have or formulas that highlight said properties when checked against these models. 